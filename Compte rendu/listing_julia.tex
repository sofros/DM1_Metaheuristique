\documentclass[10pt]{article}
\usepackage[utf8]{inputenc}
\usepackage[top=10mm, bottom=10mm, left=20mm, right=20mm]{geometry}

% individualisation des parametres de la page
\parskip8pt
\setlength{\topmargin}{-25mm}
\setlength{\textheight}{250mm}

% Sources de depart de l'adaptation de listing a julia :
% 1) https://gist.github.com/chi-feng/6589066/4cda665ff0d93b8611a0e047a6a06d6a8ecd9b4e
% 2) https://groups.google.com/forum/#!topic/julia-dev/HHjlYalHXY8

\usepackage{inconsolata} % very nice fixed-width font included with texlive-full
\usepackage[usenames,dvipsnames]{color} % more flexible names for syntax highlighting colors
\usepackage{listings}

\lstset{
basicstyle=\ttfamily, 
columns=fullflexible, % make sure to use fixed-width font, CM typewriter is NOT fixed width
numbers=left, 
numberstyle=\small\ttfamily\color{Gray},
stepnumber=1,              
numbersep=10pt, 
numberfirstline=true, 
numberblanklines=true, 
tabsize=4,
lineskip=-1.5pt,
extendedchars=true,
breaklines=true,        
keywordstyle=\color{Blue}\bfseries,
identifierstyle=, % using emph or index keywords
commentstyle=\sffamily\color{OliveGreen},
stringstyle=\color{Maroon},
showstringspaces=false,
showtabs=false,
upquote=false,
texcl=true % interpet comments as LaTeX
}

\lstdefinelanguage{julia}
{
  keywordsprefix=\@,
  morekeywords={
    exit,whos,edit,load,is,isa,isequal,typeof,tuple,ntuple,uid,hash,finalizer,convert,promote,
    subtype,typemin,typemax,realmin,realmax,sizeof,eps,promote_type,method_exists,applicable,
    invoke,dlopen,dlsym,system,error,throw,assert,new,Inf,Nan,pi,im,begin,while,for,in,return,
    break,continue,macro,quote,let,if,elseif,else,try,catch,end,bitstype,ccall,do,using,module,
    import,export,importall,baremodule,immutable,local,global,const,Bool,Int,Int8,Int16,Int32,
    Int64,Uint,Uint8,Uint16,Uint32,Uint64,Float32,Float64,Complex64,Complex128,Any,Nothing,None,
    function,type,typealias,abstract
  },
  sensitive=true,
  morecomment=[l]{\#},
%  morecomment=[s]{# =}{=#},
  morestring=[b]',
  morestring=[b]" 
}


\title{\textbf{Métaheuristiques}\\Exemple d'utilisation de \textit{listing} adapté à Julia}
\author{Nilson Toula }
\date{\today}

\begin{document}

\maketitle

%
% =================================================================================
%
\section{Code source}

\lstset{literate=
  {α}{{$\alpha$}}1 {Δ}{{$\Delta$}}1
  {á}{{\'a}}1 {é}{{\'e}}1 {í}{{\'i}}1 {ó}{{\'o}}1 {ú}{{\'u}}1
  {Á}{{\'A}}1 {É}{{\'E}}1 {Í}{{\'I}}1 {Ó}{{\'O}}1 {Ú}{{\'U}}1
  {à}{{\`a}}1 {è}{{\`e}}1 {ì}{{\`i}}1 {ò}{{\`o}}1 {ù}{{\`u}}1
  {À}{{\`A}}1 {È}{{\'E}}1 {Ì}{{\`I}}1 {Ò}{{\`O}}1 {Ù}{{\`U}}1
  {ä}{{\"a}}1 {ë}{{\"e}}1 {ï}{{\"i}}1 {ö}{{\"o}}1 {ü}{{\"u}}1
  {Ä}{{\"A}}1 {Ë}{{\"E}}1 {Ï}{{\"I}}1 {Ö}{{\"O}}1 {Ü}{{\"U}}1
  {â}{{\^a}}1 {ê}{{\^e}}1 {î}{{\^i}}1 {ô}{{\^o}}1 {û}{{\^u}}1
  {Â}{{\^A}}1 {Ê}{{\^E}}1 {Î}{{\^I}}1 {Ô}{{\^O}}1 {Û}{{\^U}}1
  {œ}{{\oe}}1 {Œ}{{\OE}}1 {æ}{{\ae}}1 {Æ}{{\AE}}1 {ß}{{\ss}}1
  {ű}{{\H{u}}}1 {Ű}{{\H{U}}}1 {ő}{{\H{o}}}1 {Ő}{{\H{O}}}1
  {ç}{{\c c}}1 {Ç}{{\c C}}1 {ø}{{\o}}1 {å}{{\r a}}1 {Å}{{\r A}}1
  {€}{{\EUR}}1 {£}{{\pounds}}1
}



\lstset{language=julia}

{
\begin{lstlisting}
# --------------------------------------------------------------------------- 
# Loading an instance of SPP (format: OR-library)

function loadSPP(fname)
    f=open(fname)
    # Lecture du nbre de contraintes (m) et de variables (n)
    m, n = parse.(Int, split(readline(f)) )
    # Lecture des n coefficients de la fonction economique et cree le vecteur d'entiers C
    C = parse.(Int, split(readline(f)) )
    # Lecture des m contraintes et reconstruction de la matrice binaire A
    A=zeros(Int, m, n)
    for i=1:m
        # Lecture du nombre d'elements non nuls sur la contrainte i (non utilise)
        readline(f)
        # Lecture des indices des elements non nuls sur la contrainte i
        for valeur in split(readline(f))
          j = parse(Int, valeur)
          A[i,j]=1
        end
    end
    close(f)
    return C, A
end

# --------------------------------------------------------------------------- 
# Construction gloutonne d'une solution admissible de SPP

function GreedyConstruction(C_in, A_in)

   # A completer...
   
    return x, z
end

# --------------------------------------------------------------------------- 
# Amelioration gloutonne par recherche locale d’une solution de SPP

function GreedyAmelioration(C_in, A_in, x_in, z_in)

   # A completer...
   
    return xbest, zbest
end

# --------------------------------------------------------------------------- 
# Exemple (compliant Julia v0.7 et ulterieur)

using Printf
fname = "Desktop/Data/pb_200rnd0100.dat"

C, A = loadSPP(fname)

@time x, z = GreedyConstruction(C, A)
@printf("z(xInit) = %d \n\n",z)

@time xbest, zbest = GreedyAmelioration(C, A, x, z)
@printf("z(xBest) = %d \n\n",zbest)
\end{lstlisting}
}
%
% =================================================================================
%
\section{Exemple d'exécution}

\begin{verbatim}
  0.299466 seconds (1.97 M allocations: 532.954 MiB, 11.09% gc time)
z(xInit) = 351 

  0.020743 seconds (9.12 k allocations: 70.348 MiB, 19.62% gc time)
z(xBest) = 357 
\end{verbatim}

\end{document}
