% =====================================================================================
% Document : rendu du DM2
% Auteur : Xavier Gandibleux , Nilson Toula, Emmanuel Rochet
% Année académique : 2019-2020


\documentclass[10pt]{article}
\usepackage[utf8]{inputenc}
\usepackage[top=10mm, bottom=10mm, left=20mm, right=20mm]{geometry}
% passe en mode large sur la page A4
\usepackage{a4wide} 

% document francisé
\usepackage[francais]{babel} 



% individualisation des parametres de la page
\parskip8pt
\setlength{\topmargin}{-25mm}
\setlength{\textheight}{250mm}

% Sources de depart de l'adaptation de listing a julia :
% 1) https://gist.github.com/chi-feng/6589066/4cda665ff0d93b8611a0e047a6a06d6a8ecd9b4e
% 2) https://groups.google.com/forum/#!topic/julia-dev/HHjlYalHXY8

\usepackage{inconsolata} % very nice fixed-width font included with texlive-full
\usepackage[usenames,dvipsnames]{color} % more flexible names for syntax highlighting colors
\usepackage{listings}

\lstset{
basicstyle=\ttfamily, 
columns=fullflexible, % make sure to use fixed-width font, CM typewriter is NOT fixed width
numbers=left, 
numberstyle=\small\ttfamily\color{Gray},
stepnumber=1,              
numbersep=10pt, 
numberfirstline=true, 
numberblanklines=true, 
tabsize=4,
lineskip=-1.5pt,
extendedchars=true,
breaklines=true,        
keywordstyle=\color{Blue}\bfseries,
identifierstyle=, % using emph or index keywords
commentstyle=\sffamily\color{OliveGreen},
stringstyle=\color{Maroon},
showstringspaces=false,
showtabs=false,
upquote=false,
texcl=true % interpet comments as LaTeX
}

\lstdefinelanguage{julia}
{
  keywordsprefix=\@,
  morekeywords={
    exit,whos,edit,load,is,isa,isequal,typeof,tuple,ntuple,uid,hash,finalizer,convert,promote,
    subtype,typemin,typemax,realmin,realmax,sizeof,eps,promote_type,method_exists,applicable,
    invoke,dlopen,dlsym,system,error,throw,assert,new,Inf,Nan,pi,im,begin,while,for,in,return,
    break,continue,macro,quote,let,if,elseif,else,try,catch,end,bitstype,ccall,do,using,module,
    import,export,importall,baremodule,immutable,local,global,const,Bool,Int,Int8,Int16,Int32,
    Int64,Uint,Uint8,Uint16,Uint32,Uint64,Float32,Float64,Complex64,Complex128,Any,Nothing,None,
    function,type,typealias,abstract
  },
  sensitive=true,
  morecomment=[l]{\#},
%  morecomment=[s]{# =}{=#},
  morestring=[b]',
  morestring=[b]" 
}


\title{\textbf{Métaheuristiques}\\Exemple d'utilisation de \textit{listing} adapté à Julia}
\author{Xavier Gandibleux }
\date{\today}




\begin{document}

\section*{Livrable du devoir maison 2 : \\ Métaheuristique GRASP, ReactiveGRASP et extensions}

%
% -----------------------------------------------------------------------------------------------------------------------------------------------------
%

\vspace{5mm}
\noindent
\fbox{
  \begin{minipage}{0.97 \textwidth}
    \begin{center}
      \vspace{1mm}
        \Large{Présentation succincte de GRASP appliqué sur le SPP}
      \vspace{1mm}
    \end{center}
  \end{minipage}
}
\vspace{2mm}

\noindent
Présenter l'algorithme mis en oeuvre. Illustrer sur un exemple didactique (poursuivre avec l'exemple pris en DM1). Présenter vos choix de mise en oeuvre.

Pour cette partie, notre programme se décompose en 
%%% Mettre ici le nombres de parties: 2 si pas de path relinking, 3 sinon
parties, la premières est une construction gloutonne prennant notament un paramètre alpha en facteur.

%
% =================================================================================
%
\section{Code source}

\lstset{literate=
  {α}{{$\alpha$}}1 {Δ}{{$\Delta$}}1
  {á}{{\'a}}1 {é}{{\'e}}1 {í}{{\'i}}1 {ó}{{\'o}}1 {ú}{{\'u}}1
  {Á}{{\'A}}1 {É}{{\'E}}1 {Í}{{\'I}}1 {Ó}{{\'O}}1 {Ú}{{\'U}}1
  {à}{{\`a}}1 {è}{{\`e}}1 {ì}{{\`i}}1 {ò}{{\`o}}1 {ù}{{\`u}}1
  {À}{{\`A}}1 {È}{{\'E}}1 {Ì}{{\`I}}1 {Ò}{{\`O}}1 {Ù}{{\`U}}1
  {ä}{{\"a}}1 {ë}{{\"e}}1 {ï}{{\"i}}1 {ö}{{\"o}}1 {ü}{{\"u}}1
  {Ä}{{\"A}}1 {Ë}{{\"E}}1 {Ï}{{\"I}}1 {Ö}{{\"O}}1 {Ü}{{\"U}}1
  {â}{{\^a}}1 {ê}{{\^e}}1 {î}{{\^i}}1 {ô}{{\^o}}1 {û}{{\^u}}1
  {Â}{{\^A}}1 {Ê}{{\^E}}1 {Î}{{\^I}}1 {Ô}{{\^O}}1 {Û}{{\^U}}1
  {œ}{{\oe}}1 {Œ}{{\OE}}1 {æ}{{\ae}}1 {Æ}{{\AE}}1 {ß}{{\ss}}1
  {ű}{{\H{u}}}1 {Ű}{{\H{U}}}1 {ő}{{\H{o}}}1 {Ő}{{\H{O}}}1
  {ç}{{\c c}}1 {Ç}{{\c C}}1 {ø}{{\o}}1 {å}{{\r a}}1 {Å}{{\r A}}1
  {€}{{\EUR}}1 {£}{{\pounds}}1
}



\lstset{language=julia}

{
\begin{lstlisting}
function GRASP(
    cost, #Une array représentant les couts, de taille n
    matrix, #une matrice de taille m*n representant les contrantes
    n, #nombre de variables
    m, #nombbre de contraintes
    alpha # le alpha que nous allons utiliser
     )

     #intitialisation des listes utilisé
     (desactive_condition, stop1, variables_actives, stop2, util, SOL) = initaliser(m,n)

    #Création de la solution
    while desactive_condition!=stop1 && variables_actives!=stop2

        util = Utilite(cost, matrix, desactive_condition, n, m , variables_actives, stop1, stop2) #Foction d'utilité

        PosCandidat::Int64 = choixcandidat(util, alpha) #Choix parmis les candidats de util

        SOL[PosCandidat] = true #On ajoute le candidat à la solution

        Desactive!(PosCandidat, matrix, desactive_condition, m, variables_actives,n) #Desactive! le candidat selectionné et les conditions où il apparait
    end

    Z = calculz(SOL,cost,n)
    return(SOL, Z, desactive_condition)

end

# =========================================================================== #
function initaliser(m,n)
    #Creating a set of lines that will be avaluated
    desactive_condition= zeros(Bool, m) #si desactive_condition[j]=0 la ligne sera évalué
    stop1= ones(Bool,m) #la une condition d'arret, s'active quand toutes les lignes ont été traité

    #Creation the an array of activated variables
    variables_actives = ones(Bool, n) #Si variables_actives[i]=1, la variable sera traité
    stop2= zeros(Bool , n) #condition d'arret, s'active quand toutes les variables sont traité

    #On crée une liste contenant nos utilités
    util = zeros(Float64, n)

    #Initialisation de la Solution
    SOL = zeros(Bool, n)

    return(desactive_condition, stop1, variables_actives, stop2, util, SOL)
end

 #=================================================================================#

#Détermine l'utilité des éléments basé sur leur nombres d'apparition dans les condition et leur valeur dans les couts
function Utilite(cost, matrix, desactive_condition, n, m , variables_actives, stop1, stop2)

    util = zeros(Float64, n) #réinitialisation du vecteur

    for j=1:n #Pour chaque variable
            if variables_actives[j] == 1 #on verifie si la variable est active
                    for i=1:m #Pour chaque condition
                        if desactive_condition[i]==0 #On verifie si la condition doit être évalué
                            K=matrix[i,j]
                            util[j]=util[j]+K
                        end
                    end

                #On divise le le cout de chqaue variable par son nombre d'appartition dans les conditions
                if util[j]!=false #On évide de divisé par 0...
                    util[j] =  cost[j] / util[j]
                else #Si la variable n'a pas été décompté, on la désactive
                    variables_actives[j] = 0
                end
            end
    end
    return util
end
#============================================================#

function Desactive!(PosCandidat, matrix, desactive_condition, m , variables_actives , n) #On désactive les lignes où est le candidat
    for i=1:m
        if matrix[i,PosCandidat] == 1
            desactive_condition[i] = 1
            for j=1:n
                if matrix[i,j] ==1
                    variables_actives[j] = 0
                end
            end
        end
    end
end

#=========================================================================#

function calculz(x,costs,m)
    z = 0
    for i in 1:m
        z+= x[i]*costs[i]
    end
    return z
end

#==========================================#

function choixcandidat(util, alpha) #Permet de construire et de choisir un candidat dans la liste de candidat restreint
    max = maximum(util)
    min = minimum(util)
    Candidat_Restreint = Float64[]
    cpt::Int=1
    for i in 1:length(util)
        if util[i]>= min+(alpha*(max-min))
            push!(Candidat_Restreint,i)
        end
    end
    prob = rand(1:length(Candidat_Restreint))
    return(Candidat_Restreint[prob])
end

\end{lstlisting}
}
%
% =================================================================================
%

%
% -----------------------------------------------------------------------------------------------------------------------------------------------------
%

\vspace{5mm}
\noindent
\fbox{
  \begin{minipage}{0.97 \textwidth}
    \begin{center}
      \vspace{1mm}
        \Large{Présentation succincte de ReactiveGRASP appliqué sur le SPP}
      \vspace{1mm}
    \end{center}
  \end{minipage}
}
\vspace{2mm}

\noindent
Présenter l'algorithme mis en oeuvre. Illustrer sur un exemple didactique (poursuivre avec l'exemple pris en DM1). Présenter vos choix de mise en oeuvre.

%
% -----------------------------------------------------------------------------------------------------------------------------------------------------
%

\vspace{5mm}
\noindent
\fbox{
  \begin{minipage}{0.97 \textwidth}
    \begin{center}
      \vspace{1mm}
        \Large{Expérimentation numérique de GRASP}
      \vspace{1mm}
    \end{center}
  \end{minipage}
}
\vspace{2mm}

\noindent
Présenter le protocole d'expérimentation (environnement matériel; budget de calcul; condition(s) d'arrêt; réglage des paramètres).

\noindent
Rapporter graphiquement vos résultats selon $\hat{z}_{min}$, $\hat{z}_{max}$, $\hat{z}_{moy}$ mesurés à intervalles réguliers (exemple de pas de 10 secondes).

\noindent
Rapporter l'étude de l'influence du paramètre $\alpha$.

\noindent
Présenter sous forme de tableau les résultats finaux obtenus pour les 10 instances sélectionnées.

%
% -----------------------------------------------------------------------------------------------------------------------------------------------------
%

\vspace{5mm}
\noindent
\fbox{
  \begin{minipage}{0.97 \textwidth}
    \begin{center}
      \vspace{1mm}
        \Large{Expérimentation numérique de ReactiveGRASP}
      \vspace{1mm}
    \end{center}
  \end{minipage}
}
\vspace{2mm}

\noindent
Présenter le protocole d'expérimentation (env. matériel; budget de calcul; condition(s) d'arrêt).

\noindent
Rapporter graphiquement vos résultats selon $\hat{z}_{min}$, $\hat{z}_{max}$, $\hat{z}_{moy}$ mesurés à intervalles réguliers (exemple de pas de 10 secondes).

\noindent
Rapporter l'apprentissage du paramètre $\alpha$ réalisé par ReactiveGRASP, les valeurs saillantes établies.

\noindent
Présenter sous forme de tableau les résultats finaux obtenus pour les 10 instances sélectionnées.

%
% -----------------------------------------------------------------------------------------------------------------------------------------------------
%

\vspace{5mm}
\noindent
\fbox{
  \begin{minipage}{0.97 \textwidth}
    \begin{center}
      \vspace{1mm}
        \Large{Eléments de contribution au bonus}
      \vspace{1mm}
    \end{center}
  \end{minipage}
}
\vspace{2mm}

\noindent
Présenter vos contributions aux aspects proposés en bonus.

%
% -----------------------------------------------------------------------------------------------------------------------------------------------------
%

\vspace{5mm}
\noindent
\fbox{
  \begin{minipage}{0.97 \textwidth}
    \begin{center}
      \vspace{1mm}
        \Large{Discussion}
      \vspace{1mm}
    \end{center}
  \end{minipage}
}
\vspace{2mm}

\noindent
Tirer des conclusions en comparant les résultats collectés avec vos deux variantes de métaheuristiques.
\noindent
Quelles sont les recommandations que vous émettez à l'issue de l'étude et avec quelle variante continuez vous l'aventure des métaheuristiques?

\vfill
\break
